\section{Cámaras como sensores}


\begin{frame}
\frametitle{Modelo de cámara}

\begin{block}{Definición - Cámara}
Una cámara es definida matemáticamente como una correspondencia entre el mundo 3D y una imagen 2D. Es decir, un mapeo entre puntos del mundo 3D $\point$ y puntos de la imagen $\imagePoint$:
\begin{equation}
\point=\begin{bmatrix}x\\
y\\
z
\end{bmatrix}\longmapsto
\imagePoint=\begin{bmatrix}u\\
v
\end{bmatrix}
\end{equation}
\end{block}

\end{frame}


\begin{frame}
\frametitle{Modelo de cámara pinhole}

\pnote{* Rayo principal o eje principal de la cámara: rayo que se origina en el centro focal \cameraCenter y es perpendicular al plano de la imagen.}

\pnote{* Punto principal: el punto donde este rayo intersecta al plano de la imagen es denominado.}

\begin{block}{Cámara pinhole}
El punto de la imagen $\imagePoint=\begin{bmatrix}u & v\end{bmatrix}^{\top}$ es determinado como la intersección entre el \emph{plano de la imagen} y el rayo que une el punto del mundo 3D $\point=\begin{bmatrix}x & y & z\end{bmatrix}^{\top}$ con el \emph{centro focal} $\cameraCenter$ de la cámara.
\end{block}

\begin{figure}[!htb]
	\centering
	\subfloat[]{\includegraphics[width=0.5\columnwidth]{./cameras/pinhole_camera_model.pdf}}%
	\hfill
\end{figure}

\end{frame}


\begin{frame}
\frametitle{Modelo de cámara pinhole}

\pnote{* Proyección central: Utilizando la propiedad de semejanza de triángulos, el punto 3D es mapeado al punto en el plano de la imagen.}

\begin{block}{Proyección central}
Asumiendo el centro focal en el origen de coordenadas y el plano de la imagen $Z=f$:

\begin{equation}
\begin{bmatrix}x\\
y\\
z
\end{bmatrix}\longmapsto\begin{bmatrix}fx/z\\
fy/z
\end{bmatrix}
\end{equation}

\end{block}

\begin{figure}[!htb]
	\centering
	\subfloat[]{\includegraphics[width=0.5\columnwidth]{./cameras/pinhole_camera_model.pdf}}%
	\hfill
	\centering
	\subfloat[]{\includegraphics[width=0.5\columnwidth]{./cameras/pinhole_camera_model2.pdf}}%
	\hfill
\end{figure}

\end{frame}


\begin{frame}
\frametitle{Modelo de cámara pinhole}

\pnote{* Lo anterior asume que el origen de coordenadas del plano de la imagen se encuentra en el punto principal. En la práctica, esto no siempre se cumple.}

\pnote{* Explicar representación en coordenadas homogéneas.}

Siendo $\begin{bmatrix}c_{u} & c_{v}\end{bmatrix}^{\top}$ la posición del punto principal.
\begin{equation}
\begin{bmatrix}x\\
y\\
z
\end{bmatrix}\longmapsto\begin{bmatrix}fx/z+c_{u}\\
fy/z+c_{v}
\end{bmatrix}
\end{equation}

En coordenadas homogéneas:

\begin{equation}
\begin{bmatrix}x\\
y\\
z\\
1
\end{bmatrix}\longmapsto\begin{bmatrix}fx+zc_{u}\\
fy+zc_{v}\\
z
\end{bmatrix}=\begin{bmatrix}f &  & c_{u} & 0\\
 & f & c_{v} & 0\\
 &  & 1 & 0
\end{bmatrix}\begin{bmatrix}x\\
y\\
z\\
1
\end{bmatrix}
\end{equation}

\begin{block}{Matriz de calibración intrínseca $\intrinsicMatrix$}
\begin{equation}
\intrinsicMatrix=\begin{bmatrix}f & 0 & c_{u}\\
0 & f & c_{v}\\
0 & 0 & 1
\end{bmatrix}
\end{equation}
\begin{equation}
\homo{\imagePoint}=\intrinsicMatrix\begin{bmatrix}\vec{{I}} & \vec{{0}}\end{bmatrix}\homo{\point}
\end{equation}
\end{block}

\end{frame}


\begin{frame}
\frametitle{Modelo de cámara pinhole}

\pnote{* Las ecuaciones anteriores deben extenderse agregando la transformación existente entre los sistemas de coordenadas de la cámara y el mundo.}

\pnote{* R es una matriz de rotación y t un vector de traslación.}

Puntos expresados en referencia al sistema de coordenadas del \textit{mundo}. La cámara no se encuentra necesariamente ubicada en el centro de este.

\begin{figure}[!htb]
	\centering
	\subfloat[]{\includegraphics[width=0.4\columnwidth]{./cameras/camera_coord_system.pdf}}%
	\hfill
\end{figure}

\begin{block}{}
\begin{equation}
\homoCameraPoint=\begin{bmatrix}\rotation & \translation\\
0 & 1
\end{bmatrix}\begin{bmatrix}x\\
y\\
z\\
1
\end{bmatrix}=\seMatrix^{\mathrm{c}\mathrm{w}}\homoWorldPoint\;.
\end{equation}
\end{block}

\end{frame}


\begin{frame}
\frametitle{Modelo de cámara pinhole}

De esta manera, es posible proyectar cualquier punto 3D
$\homoWorldPoint$ en el sistema de coordenadas del mundo al correspondiente
punto $\homo{\imagePoint}$ en el plano de la imagen mediante:

\begin{block}{Matriz de proyección \textmd{$\projectionMatrix$}}
\begin{equation}
\projectionMatrix=\intrinsicMatrix\begin{bmatrix}\rotation & \translation\end{bmatrix}
\end{equation}
\begin{equation}
\homo{\imagePoint}=\projectionMatrix\homoWorldPoint
\end{equation}
\end{block}

\end{frame}


\begin{frame}
\frametitle{Modelo de cámara estéreo}

\begin{block}{Geometría epipolar}

\pnote{* La geometría proyectiva intrínseca entre dos cámaras es conocida como geometría epipolar.}

\pnote{* Esta es independiente de la escena observada y depende únicamente de los parámetros internos geometría epipolar y las posiciones relativas de las cámaras involucradas.}

Cualquier punto 3D $\point$ en el espacio, sus proyecciones $\imagePoint$ y $\imagePoint^{\prime}$ en los planos de las imágenes y los centros focales de las cámaras, pertenecen a un mismo plano $\vec{\pi}$.
De manera análoga, los rayos re-proyectados desde $\imagePoint$ y $\imagePoint^{\prime}$ se intersectan en $\point$, son coplanares y yacen sobre $\vec{\pi}$.
\end{block}

\begin{figure}[!htb]
	\centering
	\subfloat[]{\includegraphics[width=0.5\columnwidth]{./cameras/pencil_of_planes.pdf}}%
	\hfill
	\centering
	\subfloat[]{\includegraphics[width=0.5\columnwidth]{./cameras/epipolar_plane.pdf}}%
	\hfill
\end{figure}
\end{frame}


\begin{frame}
\frametitle{Modelo de cámara estéreo}

\pnote{* La búsqueda del correspondiente al punto u no necesita cubrir el plano de la imagen completo sino que puede restringirse a la línea epipolar l.}

\pnote{* La línea epipolar es generada por la proyección del rayo, re-proyectado desde u, sobre el plano focal de la segunda imagen.}

Búsqueda de correspondencias sobre la línea epipolar.

\begin{figure}[!htb]
	\centering
	\subfloat[]{\includegraphics[width=0.5\columnwidth]{./cameras/epipolar_line.pdf}}%
	\hfill
\end{figure}
\end{frame}


\begin{frame}
\frametitle{Rectificación estéreo}

\pnote{* Luego de aplicar rectificación estéreo, la búsqueda de correspondencias entre las imágenes es reducida a una búsqueda unidimensional, sobre la misma fila.}

\begin{itemize}
\item Proyectar el par de imágenes estéreo sobre un plano de imagen común.
\item Los puntos correspondientes se encuentren alineados en la misma fila.
\end{itemize}

\begin{figure}[!htb]
	\centering
	\subfloat[]{\includegraphics[width=0.6\columnwidth]{./cameras/stereo_rectification.pdf}}%
	\hfill
\end{figure}
\end{frame}


\begin{frame}
\frametitle{Rectificación estéreo}

\pnote{* Rectificación estéreo sobre un par de imágenes del Dataset Level 7 [32]. (a) Par de imágenes estéreo original provisto por la cámara estéreo.}
\pnote{* (b) El par de imágenes estéreo luego de ser rectificadas. Las líneas rojas asocian algunos puntos correspondientes entre ambas imágenes. Notar que estos puntos se encuentran en la misma fila en ambas imágenes.}

\begin{figure}[!htb]
	\centering
	\subfloat[]{\includegraphics[width=0.6\columnwidth]{./cameras/no_rectified_stereo_images.png}}%
	\hfill
	\\
	\centering
	\subfloat[]{\includegraphics[width=0.6\columnwidth]{./cameras/rectified_stereo_images.png}}%
	\hfill
\end{figure}
\end{frame}


\begin{frame}
\frametitle{Triangulación estéreo}

\pnote{* Sea x = x y z un punto 3D cuyas coordenadas son desconocidas. Conociendo las proyecciones correspondientes, la posición del punto x puede ser derivada.}

\pnote{* Aplicando la propiedad de triángulos semejantes entre el triángulo de línea negra punteada y el triángulo de línea roja en la Figura, la siguiente ecuación puede ser formulada.}

Proyecciones correspondientes $\imagePoint_{l}=\begin{bmatrix}u_{l} & v_{l}\end{bmatrix}$ y $\imagePoint_{r}=\begin{bmatrix}u_{r} & v_{r}\end{bmatrix}$ sobre los planos focales de la imagen izquierda y derecha respectivamente,
la posición del punto $\point$ puede ser derivada.

\begin{equation}
\frac{b}{z}=\frac{(b+u_{r})-u_{l}}{z-f}
\end{equation}

\begin{figure}[!htb]
	\centering
	\subfloat[]{\includegraphics[width=0.5\columnwidth]{./cameras/stereo_triangulation.pdf}}%
	\hfill
	\centering
	\subfloat[]{\includegraphics[width=0.5\columnwidth]{./cameras/stereo_triangulation2.pdf}}%
	\hfill
\end{figure}
\end{frame}


\begin{frame}
\frametitle{Triangulación estéreo}

\pnote{* De la misma manera, utilizando propiedad de triángulos semejantes nuevamente sobre la Figura, se obtienen las restantes ecuaciones.}

\begin{equation}
\frac{x}{z}=\frac{u_{l}-c_{u}}{f}\qquad\frac{y}{z}=\frac{v_{l}-c_{v}}{f}
\end{equation}

\begin{equation}
\begin{aligned}x=\frac{(u_{l}-c_{u})z}{f}\;\qquad
y=\frac{(v_{l}-c_{v})z}{f}\;\qquad
z=\frac{bf}{u_{l}-u_{r}}\;
\end{aligned}
\end{equation}

\begin{figure}[!htb]
	\centering
	\subfloat[]{\includegraphics[width=0.5\columnwidth]{./cameras/stereo_triangulation.pdf}}%
	\hfill
	\centering
	\subfloat[]{\includegraphics[width=0.5\columnwidth]{./cameras/stereo_triangulation2.pdf}}%
	\hfill
\end{figure}
\end{frame}


\begin{frame}
\frametitle{Disparidad}

\pnote{* Cuando el valor de disparidad $d$ es cercano a $0$, pequeñas diferencias de disparidad producen un gran cambio en la profundidad del punto. En consecuencia, la reconstrucción 3D del ambiente mediante cámaras estéreo es más precisa para puntos cercanos a la cámara.}

\begin{block}{Definición - Disparidad}
La disparidad $d$ es definida como la distancia existente entre las proyecciones de las diferentes cámaras.
\begin{equation}
d=u_{l}-u_{r}=\frac{bf}{z}
\end{equation}
\end{block}

\textbf{Nota}: la profundidad $z$ es inversamente proporcional a la disparidad $d$.

\end{frame}